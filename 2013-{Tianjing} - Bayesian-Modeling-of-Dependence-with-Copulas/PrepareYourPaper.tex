\documentclass[10pt]{beamer}

%% Fonts
\usepackage{multicol}
\usepackage{mathabx}
\usepackage[scaled]{helvet}
\usepackage{lmodern}
\usepackage{eulervm}
\usefonttheme[onlymath]{serif}
\usefonttheme{professionalfonts}
\usefonttheme{structurebold}
\hypersetup{backref}
\usepackage{bm}

%% Color & Theme
\definecolor{SUblue}{RGB}{0,0,180}
\usecolortheme[RGB={0,0,180}]{structure}
\usetheme{Boadilla}
\setbeamertemplate{navigation symbols}{}
\setbeamertemplate{itemize items}[circle]
\setbeamertemplate{enumerate items}[circle]
\setbeamerfont{title}{size=\large}
\setbeamerfont{frametitle}{size=\large}
\setbeamerfont{framesubtitle}{size=\large,shape =$\color{violet}{\looparrowdownright}~$}
\setbeamercolor{title}{fg=white, bg= SUblue!75!green}
\setbeamercolor{framesubtitle}{fg=violet}
\setlength{\leftmargini}{5pt}


\title[Academic English in Statistics]{{\textbf{My two-cent experience for
      being a PhD}}}

\author[Feng Li]{\includegraphics[height=2cm]{cufelogo}\\
\vspace{0.5cm}\textbf{Feng Li}\\\texttt{feng.li@cufe.edu.cn}}
\institute[StatMath, CUFE]{\footnotesize{\textbf{School of Statistics and
      Mathematics\\ Central University of Finance and Economics}}}
\date{}


%%%%%%%%%%%%%%%%%%%%%%%%%%%%%%%%%%%%%%%%%%%%%%%%%%%%%%%%%%%%%%%%%%%%%%%%%%%%%%%
\begin{document}

%% Title page
\begin{frame}[plain]
  \titlepage
  \tiny{Revision: \today}
\end{frame}

\begin{frame}[plain]
  \begin{center}
    \texttt{It's a long long journey till I know where I am supposed to be...}
  \end{center}
\end{frame}

%% Outline
\section*{Outline}
\begin{frame}
  \frametitle{Outline}
  \addtocounter{framenumber}{-1}
  \tableofcontents
\end{frame}


\section{Read an academic paper}
\begin{frame}
  \frametitle{What do we read?}

  \begin{itemize}
  \item Many papers in our area produced each day.
  \item If you want to understand the general principle and the fundamental
    theory, a good book can be a good start.
  \item If you want known the consise results of something, read the published papers.
  \item If you want to know the \emph{stat-of-the-art} advances in your area,
    don't read the published papers. Read the working papers and participate
    research seminars.

  \item Organize your collection of papers on day one. I use
    \textbf{Name-Year-Title.pdf} naming style. Software can also help.
  \end{itemize}

\end{frame}

\begin{frame}
\frametitle{The way I read}

\begin{itemize}
\item \textbf{Bed-time-read}. Skimmed the paper and skipped some sections. Can
  account for the main ideas, but no details.

\item \textbf{Evening-read}. Read through the full paper, but skipped much of
  the details.

\item \textbf{After-lunch-read}. Read the paper. Full understanding of the main ideas and contributions. Can account for some of details and some the maths.

\item \textbf{Morning-with-a-steady-coffee-read}. Read the paper. Full
  understanding of the main ideas and contributions. Can account for most of
  the details and most of the maths.

\item \textbf{Library-with-earcuffs-read}. Ask me anything! Want an equation explained? No sweat! Heck, it feels like I even wrote this paper!

\end{itemize}

\end{frame}


% \begin{frame}
%   \frametitle{Present your results in a seminar}

%   \begin{itemize}
%   \item Think about your audience.
%   \item Think about the time.
%   \item Then think about the contents.
%   \item How many slides do you need? I usually have only one slide every
%     two-three minutes.
%   \item How much detailed information do you want to present? How much do you
%     think people can digest during your seminar.
%   \item Nervous? Practice one time at home and take some notes.
%   \item Forgot what to say next? Take some water.
%   \item When you speak, slow down please.
%   \item Please allow the audience to interrupt for questions during your
%     presentations.
%   \item Meet a difficult question from the audience?
%   \end{itemize}
% \end{frame}


\section{Find the paper}
\begin{frame}
  \frametitle{Where to download statistical papers}

  \begin{itemize}
  \item \textbf{Google Scholar}: \texttt{http://scholar.google.com}\\
    Papers with free are marked with \textbf{[PDF]} link to the right.
  \item \textbf{Working papers} with free access:
    \begin{itemize}
    \item \textbf{Social Science Research Network}:
      \texttt{http://www.ssrn.com}
    \item \textbf{Research Papers in Economics}: \texttt{http://repec.org}
    \item \textbf{e-Print archive}: \texttt{http://arxiv.org}
    \item The homepage of the authors'.

    \end{itemize}

  \item \textbf{Write to the author} directly. Don't be shy but compose your
    email carefully.

  \begin{itemize}
  \item Use your university mail account if possible.
  \item \textbf{You should always have a subject line.}
  \item Configure your email client or account with proper sender's name. \textbf{Don't
    use nicknames.}
  \item Use plain text format if possible.
  \end{itemize}


  \end{itemize}

\end{frame}

\section{The structure of a paper}

\begin{frame}
  \frametitle{The basic structure of a statistical paper}
  \begin{itemize}
  \item Title
  \item Authors and affiliations
  \item Abstract and Keywords
  \item Introduction
  \item The Model/Methodology Section
  \item Inference Section
  \item Simulations/Applications
  \item Discussion and concluding remarks
  \item References
  \item Appendix
  \end{itemize}
\end{frame}



\begin{frame}
  \frametitle{The introduction section}

  \begin{itemize}
  \item Background
    \begin{itemize}
    \item Historical/Application background
    \item Introduce necessary terminologies to make your paper be \textbf{self-contained}.
    \end{itemize}

  \item Literature reviews
    \begin{itemize}
    \item What has/hasn't been done in this topic by referencing to related
      research papers?
    \item How important of this topic?
    \item Slightly mention your idea. This will make people be interested to
      continuous to read.
    \end{itemize}
  \item Notes: You have write this section very carefully.
    \begin{itemize}
    \item Stat your opinion logically.
    \item Cite the right references.
    \end{itemize}
  \end{itemize}
\end{frame}



\begin{frame}
  \frametitle{The methodology section}
  \begin{itemize}
  \item Use a subsection to describe the model/method.
  \item The notations should follow the convention.
    \begin{itemize}
    \item e.g. $N(\mu,\sigma^2)$ for standard normal.
    \item Vector, Matrix, Lower cases, Upper cases, Greek letters.
    \item Avoid clashed notations (same notation for two purposes) throughout
      the whole paper.
    \end{itemize}
  \end{itemize}
\end{frame}

\begin{frame}
  \frametitle{The inference section}

  \begin{itemize}
  \item How to estimate the model?
  \item How to interpret the model?
  \item How to predict the model?
  \item How to perform model comparison/evaluation?
  \item Are there any competing models available?
  \end{itemize}
\end{frame}


\begin{frame}
\frametitle{Simulations}

\begin{itemize}
\item Simulation setup
  \begin{itemize}
  \item Motivate your simulations. Why and what kind of achievement do you expect?
  \item Describe clearly so other people can replicate your study.
  \end{itemize}
  \item Run the simulation (before you write down it) and summarize the results.

    \begin{itemize}
    \item Is that result as good as expected? If not, why?
    \end{itemize}

\item Think about this example: Check if the regression coefficients are biased
  via OLS in linear regression.

\end{itemize}

\end{frame}


\begin{frame}
  \frametitle{Applications}
  \begin{itemize}
  \item Describe the data background.
  \item Summarize the results by applying your model.
  \item Comparison with the competing models.
  \item Prediction.
  \end{itemize}
\end{frame}

\begin{frame}
  \frametitle{Acknowledgement}
  \begin{itemize}
  \item Show your gratitude to whom helped you in this research, supervisor,
    referees.
  \item If this paper belongs to a grant/project. You need to state that
    accordingly.
  \end{itemize}
\end{frame}

\begin{frame}
  \frametitle{Appendix}
  \begin{itemize}
  \item Only important details that are too long to be included in the main contents.
  \item Never attach computer code here.
  \end{itemize}

\end{frame}

\begin{frame}
  \frametitle{References}
  \begin{itemize}
  \item Most journals prefer author-year type references.
  \item Never include any item that is not cited in the main text.
  \item LaTeX with BibTeX solves this.
  \end{itemize}
\end{frame}




\section{Prepare your manuscripts}
\begin{frame}
  \frametitle{The general grammar guide}

  \begin{itemize}
  \item Use the present tense throughout the article.
  \item Use active voice in the article.
  \item Avoid using passive voice. It is very old fashioned.
  \item Use ``we'' instead of ``I'' or ``the author''.
  \item A statistical article is very technical and consise. Avoid long
    sentences with clauses!
  \end{itemize}
\end{frame}


\begin{frame}
  \frametitle{Capital letters}
  \begin{itemize}
  \item Don't capitalize unnecessary words.
  \item Never start a sentence with numbers, Greek letters, or mathematical
    formulas.
  \item Special words in capital
    \begin{itemize}
    \item Gaussian distribution, Poisson distribution should be in capital
      because they are named by real names. But gamma distribution is not named
      from a real name.
    \end{itemize}
  \end{itemize}
\end{frame}

\begin{frame}
  \frametitle{Common mistakes Chinese authors make}

  \begin{itemize}
  \item Singular and plural form.

  \begin{itemize}
  \item \textbf{data} is in plural \textbf{datum} is singular.
  \item \textbf{dataset} is singular.
  \item \textbf{pp.} plural; \textbf{p.} singular
  \end{itemize}

  \item \textbf{A/an/the} problem

  \item In English everything is a complete sentence.

  \end{itemize}



\end{frame}


\begin{frame}
  \frametitle{Footnotes and abbreviation}

  \begin{itemize}
  \item Footnotes should be used rarely, and should not contain mathematical
    expressions.

  \item Avoid abbreviations as much as possible. If you have to use an
    abbreviation. Declare it at the first time when it appears e.g.


\emph{... we use maximum likelihood estimation (MLE)...}

  \end{itemize}
\end{frame}


\begin{frame}
  \frametitle{Display Mathematics}

  \begin{itemize}
  \item A mathematical expression should only be numbered when it is referred in the paper.
  \item Don't write in this way

    \emph{... when $\frac{y}{x}\geq 10$ , we have...}

    but write in this way

    \emph{... when $y/x \geq 10$ , we have...}


  \item Remember that a mathematical expression is also a sentence. So the
    grammar be should be correct. Don't forget put a full stop (.) in the end
    of an equation.

  \item An equation should be followed by an explanation of all the
    abbreviations and symbols not explained before.

  \item Bold letters should only be applied to matrix or vectors

  \end{itemize}

\end{frame}


\begin{frame}
  \frametitle{Numbers}

\begin{itemize}
\item Don't start a sentence with a number.
\item Numbers like 1, 2,... in the text should be displayed as one, two,...
\item A large number should use the separator ``,'' e.g. $300,000$

\end{itemize}
\end{frame}

\begin{frame}
  \frametitle{Tables}

  \begin{itemize}
  \item The caption should be at the top of the figure.  A caption should
    comprise a brief title and a description of the
    illustration. Keep text in the illustrations themselves to a minimum but
    explain all symbols and abbreviations used.
  \item Avoid using vertical lines.
  \item Place footnotes to tables below the table body and indicate them with
    superscript lowercase letters.
  \item The data presented in tables do not duplicate results described
    elsewhere in the article.
  \end{itemize}
\end{frame}


\begin{frame}
  \frametitle{Figures}

  \begin{itemize}
  \item The caption should be at the bottom of the figure.
  \item A lot journals only accept EPS-format figures.
  \item Images supplied in color will sometimes only appear black and white in
    print due to the additional costs involved.
  \item Colors that display a clear difference may appear very similar to each
    other when converted to gray.

  \item Good practice

    \begin{itemize}
    \item Use different types of lines (dashed, dotted, solid, bold,...)
      instead of colors when constructing a plot.
    \item Crop the white margins before you embed a figure into the article to
      save space.
    \end{itemize}


  \end{itemize}
\end{frame}


\begin{frame}
  \frametitle{Citations}

  \begin{itemize}
  \item Author style
  \begin{itemize}
  \item Only one author
  \item Less than four authors
  \item More than four authors
  \end{itemize}

\item Cite in text and cite in parentheses

\item Singular or plural when meet ``et al''

\item Tense

\item Use BibTeX with LaTeX with make life a lot easier.

  \end{itemize}

\end{frame}

\begin{frame}
  \frametitle{References}

  \begin{itemize}
  \item A reference should only show up when it is cited in the paper.
  \item One should follow the standard style of References. The most common one
    is the author-year style, e.g.



    Beal, M. J., Falciani, F., Ghahramani, Z., Rangel, C., \& Wild,
    D. L. (2005). A Bayesian approach to reconstructing genetic regulatory
    networks with hidden factors. \emph{Bioinformatics}, 21(3), 349-356.


  \end{itemize}

\end{frame}


\section{Submit your paper}
\begin{frame}
  \frametitle{Before your submission}

  \begin{itemize}
  \item Ask your colleagues about the quality of your paper. You have do a few
    seminars too.
  \item You should always have your working paper available to the public
    before your submit it to the journal.
  \item Find the proper journal that matches your theme.
  \item Do a little survey just to check out the editor's tastes.
  \item Never submit one paper to two journals simultaneously.


  \end{itemize}
\end{frame}


\begin{frame}
  \frametitle{Submit your manuscript}

  \begin{itemize}

  \item Proof read and check your language.

  \item Modify your paper (mostly the style) by following the author's guide
    provided by the journal.

  \item Sometimes you have to write a cover letter to the editor too.
  \item Then just wait.
  \end{itemize}

\end{frame}


\begin{frame}
  \frametitle{Revise your paper}
  \begin{itemize}
  \item Most referees  are unlikely to reject a proper written paper directly.
  \item Most time you will get \textbf{revise and resubmit} decision.
  \item Revise it \textbf{strictly} by following the revision reports.
  \item Mind your words and do respect the referees.
  \item It is very important to resubmit it on time.
  \item Even if you got rejected, revise it first and resubmit to another journal.
  \item The revision cycle usually looks like this:

    \textbf{Major revision} $\Rightarrow$ \textbf{Minor revision} $\Rightarrow$
    \textbf{Accepted} $\Rightarrow$ \textbf{Uncorrected proof} $\Rightarrow$
    \textbf{Corrected proof} $\Rightarrow$ \textbf{Article is online}
    $\Rightarrow$ \textbf{Finally printed}.
  \end{itemize}

\end{frame}


\begin{frame}[plain]
  \begin{center}
    \texttt{Then your paper is published. You start a new paper from scratch.
    }
\end{center}
\end{frame}

\begin{frame}[plain]
  \addtocounter{framenumber}{-1}
  \begin{center}
    {\color{SUblue} \textbf{\Huge Thank you!}}
    \vspace{1cm}

    {\texttt{\textbf{\url{feng.li@cufe.edu.cn}}}}

    \vspace{1cm}

    {\texttt{\textbf{\url{http://feng.li/}}}}

  \end{center}
\end{frame}

\end{document}
